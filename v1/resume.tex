%------------------------------------------
% Resume for entry level Software Engineering roles
% Author : Abhishek Jha
% Github : https://github.com/abhijhacodes/latex-resume
% Website: https://abhijha.in
%------------------------------------------

\documentclass[a4paper,11pt]{article}

% --------------- PACKAGES -----------------------
\usepackage{latexsym}
\usepackage[empty]{fullpage}
\usepackage{titlesec}
\usepackage{marvosym}
\usepackage[usenames,dvipsnames]{color}
\usepackage{verbatim}
\usepackage{enumitem}
\usepackage[hidelinks]{hyperref}
\usepackage{fancyhdr}
\usepackage[english]{babel}
\usepackage{tabularx}
\usepackage{fontawesome}
%% Font Definitions
\usepackage[sfdefault]{universalis}
\usepackage[T1]{fontenc}
\usepackage{microtype}

% ---------- RESUME FORMAT ADJUSTMENTS -------------
\addtolength{\oddsidemargin}{-0.5in}
\addtolength{\evensidemargin}{-0.5in}
\addtolength{\textwidth}{1.0in}
\addtolength{\topmargin}{-.7in}
\addtolength{\textheight}{1.0in}

\urlstyle{same}

\raggedbottom
\raggedright
\setlength{\tabcolsep}{0in}

\titleformat{\section}{
  \vspace{-4pt}\raggedright\large
}{}{0em}{}[\color{black}\titlerule \vspace{-5pt}]

% \pdfglyphtounicode=1

\newcommand{\resumeItem}[1]{
  \item\small{
    {#1 \vspace{-2pt}}
  }
}

\newcommand{\resumeItemNoBullet}[1]{
  \item[]\small{
    {#1 \vspace{-2pt}}
  }
}

\newcommand{\resumeSubheading}[4]{
  \vspace{-2pt}\item
    \begin{tabular*}{1\textwidth}[t]{l@{\extracolsep{\fill}}r}
      \textbf{#1} & #2 \\
      \textit{\small#3} & \textit{\small #4} \\
    \end{tabular*}\vspace{-8pt}
}

\newcommand{\resumeSubheadingNoItalics}[4]{
  \vspace{-2pt}\item
    \begin{tabular*}{1\textwidth}[t]{l@{\extracolsep{\fill}}r}
      \textbf{#1} & #2 \\
      \small#3 & \textit{\small #4} \\
    \end{tabular*}\vspace{-2pt}
}

\newcommand{\resumeSubSubheading}[2]{
    \item
    \begin{tabular*}{0.97\textwidth}{l@{\extracolsep{\fill}}r}
      \textit{\small#1} & \textit{\small #2} \\
    \end{tabular*}\vspace{-7pt}
}

\newcommand{\resumeProjectHeading}[2]{
    \item
    \begin{tabular*}{1\textwidth}{l@{\extracolsep{\fill}}r}
      \small#1 & #2 \\
    \end{tabular*}\vspace{-7pt}
}

\newcommand{\resumeSubItem}[1]{\resumeItem{#1}\vspace{-4pt}}

\renewcommand\labelitemii{$\vcenter{\hbox{\tiny$\bullet$}}$}

\newcommand{\resumeSubHeadingListStart}{\begin{itemize}[leftmargin=0in, label={}]}
\newcommand{\resumeSubHeadingListEnd}{\end{itemize}}
\newcommand{\resumeItemListStart}{\begin{itemize}[leftmargin=0.15in, nosep]}
\newcommand{\resumeItemListEnd}{\end{itemize}\vspace{-2pt}}


%-----------------RESUME STARTS HERE----------------------
% The resume has Education, Experience, Projects, Achievements, Technical Skills, Coding Profiles sections
% You can keep which all you want and in which order, comment the ones you don't need
% In ordering your sections, keep the ones first which are your best selling points

\begin{document}

\noindent % Prevents indentation
\begin{tabularx}{\textwidth}{@{} X c >{\raggedleft\arraybackslash}X @{}}
    \begin{tabular}{@{}l}
        \href{mailto:luispre99@gmail.com}{luispre99@gmail.com} \\
        (+52) 3317933872
    \end{tabular}
    & 
    \begin{tabular}[C]{@{}c}
        \bfseries\huge \textls[0]{Luis Enrique Preciado Muñiz} \\
    \end{tabular}
    &
    \begin{tabular}{@{}r}
        Guadalajara, México \\
        C.P 45200
    \end{tabular}
\end{tabularx}
\vspace{-12pt}
\begin{center}
    \href{https://lepcodes.vercel.app/}{\textcolor{BlueViolet}{\faGlobe\enspace \textbf{lepcodes.vercel.app}}} $|$  
    \href{https://www.linkedin.com/in/luispreciado/}{\textcolor{BlueViolet}{\faLinkedin\enspace \textbf{linkedin.com/in/luispreciado}}} $|$
    \href{https://github.com/lepcodes}{\textcolor{BlueViolet}{\faGithub\enspace \textbf{github.com/lepcodes}}}
\end{center}

% \vspace{10pt}
% \begin{center}
%     \textbf{\huge Luis Enrique Preciado Muñiz} \\ \vspace{8pt}
%     \faMapMarker\enspace \textbf{Guadalajara, México} $|$
%     \href{luispre99@gmail.com}{\textcolor{BlueViolet}{\faEnvelopeO\enspace \textbf{luispre99@gmail.com}}} $|$  
%     \href{https://www.linkedin.com/in/luispreciado/}{\textcolor{BlueViolet}{\faLinkedin\enspace \textbf{LinkedIn}}} $|$
%     \href{https://github.com/lepcodes}{\textcolor{BlueViolet}{\faGithub\enspace \textbf{Github}}} $|$
%     \small {\faMobile\enspace \textbf{+52 3317933872}}
% \end{center}
\vspace{-12pt}

%------------PROFILE------------
\rule{\textwidth}{0.1pt}

\vspace{2pt}
I'm a passionate and motivated individual with a strong interest in programming, automation, and robotics. Nearing the completion of my Robotics Engineering degree, I am seeking a graduate position to further enhance my knowledge and make meaningful contributions to the industry. With over a year of experience as an intern in the automotive industry, I am committed to continuous learning and professional growth.
\vspace{-8pt}

%-----------EDUCATION-----------
% Write only your two highest formal educational qualifications
\section{\Large{Education}}
  \resumeSubHeadingListStart
    \resumeSubheading
      {University of Guadalajara}{Guadalajara, México}
      {Master's Degree in Machine Learning and AI -- \textbf{CGPA: 9.5}}{January 2025 -- Currently}
    \resumeSubheading
      {University of Guadalajara}{Guadalajara, México}
      {Robotics Engineering -- \textbf{CGPA: 9.8}}{January 2020 -- June 2024}
    \resumeSubheading
      {CETI (Technical and Industrial Teaching Center)}{Guadalajara, México}
      {Automatic Control Technologist -- \textbf{Percentage: 88\%}}{July 2017 -- April 2019}
  \resumeSubHeadingListEnd
\vspace{-11pt}

%-----------EXPERIENCE-----------
\section{\Large{Experience}}
  \resumeSubHeadingListStart
    \resumeSubheadingNoItalics
      {Software Engineer Intern}{August 2023 -- July 2024}
      {\href{https://bosch.com.mx}{\textcolor{BlueViolet}{\textbf{\large{Robert Bosch México}}}}}{Guadalajara, México (Hybrid)}
      \resumeItemListStart
        \resumeItem{Designed and implemented a Python and Rasa framework-based Chatbot to facilitate new developer induction and provide troubleshooting assistance, complete with a local SQL database for response management, a ChatGPT-based model API for enhanced query handling, and Github for version control.}
        \resumeItem{Automating report generation for fault-mapping between customer defined fault application and ECU's  real monitors using Python scripts, leveraging Pandas, Anaconda, and fuzzy logic libraries, to ensure accurate tracking of monitor-fault mappings and to guarantee software quality deliverables.}
        \resumeItem{Developed scripts to cross-reference client requirements with ECU header files, identifying potential inconsistencies and streamlining the verification process.}
        \resumeItemNoBullet{\textbf{Key Metrics:} Enhanced quality of deliverables and improved efficiency by 40\%, reducing inconsistencies by 90\%.}
      \resumeItemListEnd
    \vspace{-2pt}
    \resumeSubheadingNoItalics
      {Robotics Intern}{Jun. 2022 - Aug. 2023}
      {\href{https://www.cucei.udg.mx/carreras/robotica/es/laboratorios/ciber-fisicos}{\textcolor{BlueViolet}{\textbf{\large{Intelligent Systems Laboratory}}}}}{Guadalajara, México}
      \resumeItemListStart
        \resumeItem{Supported the development of research on consensus algorithms for the Turtlebot robot platform and other holonomic robots using motion capture technology, optical tracking (OptiTrack), and ROS drivers on a Linux environment.}
        \resumeItem{Extensively worked with Linux in the assembly, programming, and testing of UAV units, as well as the implementation of monocular and stereoscopic visual-inertial odometry algorithms focused on indoor flights.}
        \resumeItem{Performed PCB and circuit design to synchronize measurements from optical and inertial sensors at the hardware level, which was required for the implementation of visual odometry.}
        \resumeItemNoBullet{\textbf{Key Metrics:} Modified existing ROS/ROS2 C++ drivers in a Linux environment to synchronize visual-inertial measurements and developed new drivers to control holonomic robots and UAV kinematics}
      \resumeItemListEnd
  \resumeSubHeadingListEnd

\vspace{-18pt}
%-----------PROJECTS-----------
\section{\Large{Projects}}
    \resumeSubHeadingListStart
      \resumeProjectHeading
          {\textbf{\large{Chat React Component}} $|$ \emph{\href{https://google.com/}{\textcolor{BlueViolet}{\textbf{Live}}}} $|$ \emph{\href{https://github.com/}{\textcolor{BlueViolet}{\textbf{Github}}}}}{React JS, Express, MongoDB, Chakra UI, Stripe}
          \resumeItemListStart
            \resumeItem{Give abstract overview of what this project is about and what it does}
            \resumeItem{Keep \textbf{highlighting} the key poitns and include numbers if you can, they help}
            \resumeItem{Explain about major features of your project, don't write something you didn't make}
            \resumeItem{\textbf{Features:} Search, Rest APIs, Realtime database, High performance, Responsive UI, etc. }
          \resumeItemListEnd
          
    \resumeProjectHeading
          {\textbf{\large{Omni-Driver}} $|$ \emph{\href{https://google.com/}{\textcolor{BlueViolet}{\textbf{Live}}}} $|$ \emph{\href{https://github.com/}{\textcolor{BlueViolet}{\textbf{Github}}}}}{React JS, Firebase, Tailwind CSS}
          \resumeItemListStart
            \resumeItem{Don't overflood your resume with projects, just keep the projects you are proud of and you can explain it in detail to interviewer}
            \resumeItem{If you have a lot of projects you can try to keep the ones depending on company you are applying to (E commerce for Amazon, Video Streaming for Netflix, etc.)}
            \resumeItem{\textbf{Features:} Search, Rest APIs, Realtime database, High performance, Responsive UI, etc. }
          \resumeItemListEnd
          
    \resumeSubHeadingListEnd

%-----------TECHNICAL SKILLS-----------
% Don't fake this section, as interviewer might screw you in it

\section{\Large{Technical Skills}}
 \begin{itemize}[leftmargin=0.15in, label={}]
    \small{\item{
     \textbf{Languages}{: Python, Javascript, Typescript, HTML, CSS, C/C++} \\
     \textbf{Frameworks}{: Tensorflow, Pandas, React JS, Next JS, Node, Express JS, FastAPI, Styled components, Tailwind} \\
     \textbf{Databases}{: SQLite, PostgresQL, Supabase} \\
    }}
 \end{itemize}


% --------CODING PROFILES--------------
% keep only your best coding profiles here
% add \\ after each 3rd profile to add a new line, if you add more profiles (which you shouldn't)
% \section{Coding Profiles}
% \renewcommand{\arraystretch}{0}
% \begin{tabular}{p{0.33\textwidth}p{0.33\textwidth}p{0.33\textwidth}}
%   \begin{itemize}[leftmargin=*, topsep=0pt]
%     \item LeetCode - \href{https://leetcode.com/username}{\textbf{\textcolor{BlueViolet}{username}}}
%   \end{itemize} &
%   \begin{itemize}[leftmargin=*, topsep=0pt]
%     \item Github - \href{https://github.com/username}{\textbf{\textcolor{BlueViolet}{username}}}
%   \end{itemize} &
%   \begin{itemize}[leftmargin=*, topsep=0pt]
%     \item HackerRank - \href{https://www.hackerrank.com/username}{\textbf{\textcolor{BlueViolet}{username}}}
%   \end{itemize} \\
% \end{tabular}


% -----------ACHIEVEMENTS/OPEN-SOURCE/RESEARCH-----------
% Keep achievements as per the company you are applying to
% If you have achievements in competitive programming, add it, it is useful when applying to product based companies
% You should give more priority to hackathons' and open source's achievements when applying for startups or abroad companies
% \section{Achievements}
% \resumeItemListStart
%     \resumeItem{Keep your relevant achievements only}
%     \resumeItem{Don't write about the art competitions you won in school or what was your rank in board exams, no one cares about that}
%     \resumeItem{Add \href{https://google.com/}{\textcolor{BlueViolet}{\textbf{link}}} to your achievements if you have, as a source of proof}
%     \resumeItem{\textbf{Highlight} the key points you want the other person to notice}
%     \resumeItem{Don't write that you were CR of your class in achievements, remember you are applying for a software engineering role}
% \resumeItemListEnd

%-------------------------------------------
\end{document}